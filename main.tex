%% For NIH grant application %%
%% July 2010 %%
%% Tatsuki Koyama %%

\documentclass[11pt]{article}  %11 or 12 pt

% Packages some may find useful.
% \usepackage{graphicx,epsf,psfig,pstricks,subfigure,psfrag,rotating}

% New NIH specifications
% font = Arial, Helvetica, Palatino Linotype, or Georgia typeface
% font size 11 or larger
% margin = at least one-half inch
% Use at least one-half inch margins for all pages.  
% No information should appear in the margins, including the PI's name and page numbers.

\renewcommand{\rmdefault}{phv} % Arial
\renewcommand{\sfdefault}{phv} % Arial

\usepackage[width=7.5in, height=10.0in, head=0.0in, foot=0.0in, headsep=0.0in]{geometry}
%% This controls margins.  Can't go over width=7.5in, height=10.0in.  
%% top-bottom margins = (11-height)/2  left-right margins = (8.5-width)/2

\usepackage{setspace} % useful in changing vertical spacing temporarily.

\usepackage[superscript,biblabel]{cite}
%\usepackage{natbib} % more control over how references appear within text.
%\bibpunct{[}{]}{,}{n}{}{} % like so.
%% use \citep{ref} instead of \cite{ref} in text.

\usepackage{sectsty} % can change font, size of the section headings.  
\sectionfont      {\fontsize{12pt}{3}\usefont{OT1}{phv}{b}{sc}\selectfont}
\subsectionfont   {\fontsize{11pt}{3}\usefont{OT1}{phv}{b}{n}\selectfont}
\subsubsectionfont{\fontsize{11pt}{3}\usefont{OT1}{phv}{m}{n}\selectfont}

%\usepackage{titlesec}
%\titleformat{\subsection}[runin]{}{}{}{}[]


%\renewcommand{\thesection}{\Alph{section}} % so that section headings use A B C instead 1 2 3
%\renewcommand{\baselinestretch}{1}

\renewcommand\refname{\section*{Literature Cited} \vspace{-1em}} 
%% This changes ``Reference'' to ``Literature Cited''.  

\newcommand{\inden}[1]{\mbox{} \hspace{#1} } % Force horizontal spaces.  

%% NO INDENTATION %%
\newlength\tindent
\setlength{\tindent}{\parindent}
\setlength{\parindent}{0pt}
\renewcommand{\indent}{\hspace*{\tindent}}

% -- % -- % -- % -- % -- %
% -- % -- % -- % -- % -- %
\begin{document}
\pagestyle{empty}











% -- % -- % -- % -- % -- %
% -- % -- % -- % -- % -- %
\section*{Specific Aims}
\hspace{10mm}Polygenic Risk Scores (PRS) are of significant interest to clinical and statistical geneticists for their potential to accurately capture individuals' genetic liability for developing disease. PRS are computed from effect size estimates ($\beta$) in Genome-Wide Association Studies (GWAS) \cite{choi_tutorial_2020}. However, GWAS are subject to significant bias \cite{dudbridge_power_2013} as (1) phenotypic labeling of cases and controls often relies on broad, subjective classifications and (2) the majority of available genomic studies focus on populations with large European ancestry. Therefore, there exists a \textbf{critical need to develop a model of PRS that accounts for these biases in order to improve predictive power}.

\hspace{10mm}GWAS rely on clinical descriptions of phenotypes to distinguish cases from controls. However, large-scale genomic analyses have shown that many clinical traits classified as arising from a single etiology may in fact be comprised of multiple subtypes, each with a unique genetic architecture \cite{nicolau_topology_2011}. PRS computed by homogenizing these phenotypic subtypes would fail to accurately explain the variance in phenotype, \textbf{as multiple distinct phenotypes are essentially treated as one}. Additionally, previous work has shown that analysis of diverse, admixed populations improves GWAS detection of complex trait associated alleles \cite{wojcik_genetic_2019}. Additionally, PRS fail to accurately transfer across ancestry groups. PRS in non-European populations are most often calculated using the same European GWAS data and neglect to account for the distinctions in demographic history that affect the data \cite{wojcik_genetic_2019}, for instance: patterns of linkage disequilibrium, natural selection and environmental effects. These population genetic factors introduce unique variants that contribute to the development of distinct disease etiologies unaccounted for in standard PRS methods. 

\hspace{10mm}We provide a novel approach for improving the predictive power of PRS: $\Psi$PRS (\textit{psi-PRS}) \textbf{P}henotypic \textbf{S}ubtype \textbf{I}nformed \textbf{P}olygenic \textbf{R}isk \textbf{S}cores. First, we employ standard clustering and classification algorithms to identify trait subtypes using whole-genome sequencing resources such as UK BioBank and Mount Sinai BioMe \cite{bycroft_uk_2018}. We cross-validate inferred subtypes with previously identified subtypes (e.g. cardiovascular disease) and use standard heterogeneity tests for statistical estimates of validity \cite{dahl_robust_2020}. Then, we compute PRS from the relabeled phenotypic subtypes using the gold standard software PRSice \cite{choi_prsice-2_2019}. Our $\Psi$PRS account for unique trait etiologies lost in standard phenotype labeling. We consider subsets of standard cases and controls given different genetic architectures of clinically similar disease states. 

\textbf{\textit{Hypothesis.}} By constructing PRS for specific phenotypic subtypes instead, we expect improved predictive power of these scores and improved transferability of scores across diverse populations with significant non-European ancestry.

% -- % -- % -- % -- % -- %
\subsection*{Aim 1. Develop $\Psi$PRS model for complex disease.}
% \inden{2em}   A.1.1: sub aim 1\\
\underline{Subaim 1a.} Identify subtypes of cardiovascular disease and heart failure phenotypes in UK BioBank and Mount Sinai BioMe using standard clustering and classification algorithms. \underline{Subaim 1b.} Validate inferred phenotypic subtypes using SNP-level \cite{dahl_reverse_2019} and polygenic heterogeneity tests \cite{dahl_robust_2020} as well as using known sub-classifications of individual complex traits \cite{li_identification_2015}. \underline{Subaim 1c.} Compare $\Psi$PRS explained phenotypic variance to standard genome-wide PRS.

% -- % -- % -- % -- % -- %
\subsection*{Aim 2. Investigate reduced PRS transferability in non-European ancestry populations.}
\underline{Subaim 2a.} Model multi-way admixture of global sub-populations using simulations in standard \textbf{SLiM} population genetic software \cite{haller_slim_2019} and genotypes from the 1000 Genomes Project reference panels \cite{noauthor_global_2015}. \underline{Subaim 2b.} Compute ancestry-specific PRS with data from multi-ethnic populations in model along with Mount Sinai BioMe and UK BioBank to investigate causes of reduced transferability: differences in associated variant frequencies across populations, missing effects of population-specific rare variants and discrepancies in phenotype classification. 

\subsection*{Impact}
This proposal will deliver a novel approach for leveraging phenotypic subtypes in computing PRS. $\Psi$PRS considers unique genetic architectures in phenotypic subtypes to improve predictive power of PRS. We propose that missing phenotypic subtypes contribute to the limited transferability of standard PRS. Our work addresses the \textbf{essential need} for PRS to accurately capture genetic liability for diverse populations.         

%-----%
\newpage
\section*{Research Strategy}

\subsection*{Significance}
% [1] https://www.ncbi.nlm.nih.gov/pmc/articles/PMC7016826/
% [2] https://jamanetwork.com/journals/jama/fullarticle/195451
% [3] https://www.ahajournals.org/doi/10.1161/01.RES.0000129254.25507.d6?url_ver=Z39.88-2003&rfr_id=ori%3Arid%3Acrossref.org&rfr_dat=cr_pub++0pubmed&
% [4] https://pubmed.ncbi.nlm.nih.gov/30007560/
% [5] https://pubmed.ncbi.nlm.nih.gov/30788622/


% [A] https://www.pnas.org/content/108/17/7265
% [B] https://www.ncbi.nlm.nih.gov/pmc/articles/PMC2995704/
% [C] https://journal.chestnet.org/article/S0012-3692(18)30164-8/fulltext
% [D] https://pubmed.ncbi.nlm.nih.gov/19744292/
% [E] https://link.springer.com/article/10.1007/s10741-019-09828-8
% [F] https://www.ncbi.nlm.nih.gov/pmc/articles/PMC5870132/



\hspace{10mm}Improving Polygenic Risk Scores (PRS) will have a widespread impact on biomedical research and personalized medicine. Genome-Wide Association Studies (GWAS) have elucidated the genetic bases for numerous complex diseases. Countless associations have been made between genotypes and phenotypes of interest to researchers and clinicians. GWAS report summary statistics including effect size estimates ($\beta$) for implicated variants and phenotypes of interest. PRS aim to capture an individual's genetic liability for disease based on these GWAS effect size estimates. Predicting individual disease risk from genetic information provides a powerful opportunity to design personalized treatment plans and robust preventative care. Assessing genetic risk for various conditions enables early monitoring which in many cases can prove invaluable long before disease manifestation. However, before broad clinical adoption of PRS is attainable, researchers must ensure high predictive accuracy and usability across populations of diverse genetic ancestry.

\hspace{10mm}PRS have recently shown significant utility in predicting cardiovascular disease risk \cite{sun_polygenic_2021}. Additionally, several recent trials of PRS have assessed their usefulness in clinical settings, particularly for coronary artery disease (CAD) and heart failure \cite{levin_michael_g_polygenic_2020}. Heart failure is the top cause of hospitalization across the globe \cite{simmonds_cellular_2020}. Clinically, heart failure is stratified according to a patient’s left ventricular ejection fraction (EF). Heart failure with ejection fraction of less than 40\% is termed HFrEF (heart failure with reduced ejection fraction), while heart failure with ejection fraction greater than 50\% is termed HFpEF (heart failure with preserved ejection fraction). Epidemiologically, HFrEF incidence has steadily declined in the US while HFpEF incidence continues to rise \cite{tsao_temporal_2018}.

\hspace{10mm}HFrEF and HFpEF are thought to arise from different pathophysiological mechanisms \cite{kitzman_pathophysiological_2002}. HFrEF is closely associated with coronary artery disease (CAD). CAD is thought to cause ischemia in cardiomyocyte and therefore provides a mechanistic account of the reduced ejection fraction observed clinically in HFrEF. The mechanistic understanding of HFrEF has better enabled identification of risk factors, motivated therapeutic development, and informed chronic care management \cite{kass_david_a_what_2004}. 

\hspace{10mm}While the clinical classification of heart failure according to its ejection fraction has improved management of HFrEF, the HFpEF class seems to be much more heterogeneous \cite{kass_david_a_what_2004}. Whereas HFrEF is closely associated with CAD, HFpEF is more broadly associated with chronic disease such as obesity, hypertension, type 2 diabetes mellitus, and renal insufficiency \cite{ergatoudes_non-cardiac_2019}. A better understanding of risk factors and mechanisms underlying HFpEF remains an important initiative of public health.

\hspace{10mm}The generation of massive clinical datasets, such as the repositories of clinical genomic information (e.g. UK BioBank and Mount Sinai BioMe), has demanded the generation of new computational approaches for analyses. Such analyses can leverage rich patterns in high-dimensional spaces to reveal biological insights \cite{nicolau_topology_2011-1}. In the case of genomic repositories, one major consideration involves the resolution of phenotype classifications, or clinical labels assigned to each sample. 

\hspace{10mm}Phenotype classifications in GWAS and PRS calculation rely on clinical labels from health records. There exist several common standardized labeling systems for phenotypes in genomic data, with the most common being from the International Statistical Classification of Diseases and Related Health Problems (ICD). ICD is important for standardizing diagnoses in epidemiologic analyses and billing \cite{steindel_international_2010}. ICD-10 is the 10th version of the ICD and the version currently used for clinical and research purposes. While the classification schemes provided by the ICD are often used clinically, their role in research has been contested \cite{weiner_point_2018, cox_good_2009}. In the case of HFpEF, referred to as “diastolic heart failure” in ICD-10-CM, the scheme biases what may be a heterogenous group of diseases arising through independent pathophysiological mechanisms \cite{severino_structural_2020}. When performing genomic analyses such as polygenic risk scores, classification of a disease is confounded when the disease label includes multiple mechanisms. This could be one explanation for the lack of genotypic correlations identified with HFpEF relative to HFrEF \cite{andersson_association_2018}.

\hspace{10mm}To address the potential for multiple heart failure sub-types being encompassed within the ICD-10-CM label of “diastolic heart failure”, we propose to perform unsupervised clustering of HFpEF classified patients according to single nucleotide polymorphisms (SNPs) in whole genome sequences. We will then calculate polygenic risk scores of these individual clusters, or assumed HFpEF sub-types, and compare them to polygenic risk scores calculated for the entire group of HFpEF patients. By introducing a clustering step we aim to isolate subtypes of HFpEF that lead to more predictive polygenic risk scores than those performed across the entire HFpEF classification.

\hspace{10mm}Further, despite great promise for PRS as clinical screening tools, standard PRS are subject to significant bias and have limited reliability across populations of differing ancestries \cite{majara_low_2021}. This is predominantly a consequence of the missing diversity in human genetic studies \cite{sirugo_missing_2019}. The majority of genetic association studies of disease rely on data collected from populations of European ancestry. With such a limited representation of global human genetic diversity, GWAS and PRS are largely underpowered to capture risk variants that are unique to non-European populations. Several approaches to improving PRS transferability have been developed. Some involve statistical approaches to generalizing PRS constructed from European data using data from non-European and admixed populations \cite{grinde_generalizing_2019}. Other approaches involve computing PRS directly from non-European populations \cite{martin_critical_2018}. With many validated by incorporating additional data from non-European populations directly \cite{cavazos_inclusion_2021}. In order to optimize the utility of PRS, genetic studies of disease must consider contributions from populations of diverse genetic ancestry. 

\hspace{10mm}In our aims, we propose to develop the $\Psi$PRS model to improve the predictive power of PRS by maximally leveraging classification methods to identify phenotypic subtypes of disease and computing PRS with these subtypes. This approach addresses the heterogeneity in disease that leads to similar phenotypes with different underlying genetic etiologies being classified as a single phenotype. We also propose to investigate reduced PRS transferability in populations of diverse ancestry by assessing differences in population structure and phenotype genetic architecture that can influence PRS. 





\subsection*{Approach}

\textbf{\textit{Background.}} Standard PRS are calculated as a sum of an individual's genotyped variants weighted by the GWAS effect size estimates for a given trait \cite{choi_tutorial_2020}. An estimated PRS for an individual $\hat{S}$ is computed as:

$$
\hat{S} = \sum_{i=1}^{N} \beta_i X_i
$$

Where each individual has $N$ marker variants, or variants found to be associated with the trait, genotypes for the specified trait with each genotype denoted as $X_i$ and effect size $\beta_i$. This standard genome-wide PRS assumes independent, additive contributions of variants to an individual's genetic risk for the trait. Diseases with multiple variable genetic etiologies i.e. subtypes which arise from unique sets of risk variants, are merged into a single set of effect sizes. Our analyses will expand upon the standard PRS calculation and incorporate data with ICD-10 labeled phenotypes. 

\textbf{\textit{The UK Biobank}} is a biomedical database that contains genetic and Electronic Health Record (EHR) data from 500 million UK volunteers. It contains whole exome sequencing from nearly 50k volunteers along with rich phenotype data including nearly 6000 participants with reported heart failure. It is the largest publicly available genetic and phenotypic database in the world \cite{sudlow_uk_2015}. The resource will be invaluable in developing and validating methods to improve PRS predictive power via phenotype subtyping.

\textbf{\textit{The Mount Sinai Bio\texit{Me} Biobank}} is one of the most diverse EHR-linked genetic data collections in the world \cite{belbin_toward_2021}. It includes information from more than 50k participants and nearly 30k of those include genotype and whole exome sequencing. Participants volunteer from the nearly 2 million patients served by Mount Sinai in New York City. The resource represents significant ethnic, clinical and socioeconomic diversity that will be invaluable in assessing the transferability of PRS across diverse populations.   

\hspace{10mm}Genomic repositories such as the UK Biobank and Mount Sinai BioMe Biobank often employ standardized ICD labeling for diagnoses. These labels, constructed for billing and epidemiologic analyses, are not always reflective of pathophysiology and present a bias when applied to classify cohorts for genomic analyses \cite{weiner_point_2018, cox_good_2009}. Controversy exists over whether the ICD-10-CM label of “diastolic heart failure,” also known as HFpEF, encompasses several distinct forms of heart failure \cite{severino_structural_2020}. Performing a polygenic risk score on all patients diagnosed under “HFpEF” may be averaging over several different pathophysiological mechanisms of heart failure which preserve ejection fraction, thus reducing the predictive power of genotypic association with any single pathophysiological mechanism \cite{andersson_association_2018}.

\hspace{10mm}Clustering is an unsupervised machine learning method which has demonstrated remarkable success in handling high-dimensional genomic datasets \cite{hibar_genetic_2013,xu_model-based_2019, narita_clustering_2020}. Here we aim to leverage a clustering approach to stratify the ICD-10-CM classification of “diastolic heart failure”, also known as HFpEF, patients into clusters according to genotypic differences and then calculate polygenic risk scores for each cluster of HFpEF patients. In doing so, we attempt to stratify the heterogenous HFpEF population into clusters that better reflect distinct pathophysiological mechanisms of heart failure with preserved ejection fraction and in turn better correlate with genotypic markers of disease.

%% [AA] https://www.nature.com/articles/s41398-020-00951-x
%% [AB] https://www.nature.com/articles/s41598-019-50229-6
%% [AC] https://www.ncbi.nlm.nih.gov/pmc/articles/PMC4024454/



% -- % -- % -- % -- % -- %
\subsection*{Aim 1. Develop $\Psi$PRS model for complex disease.}

\indent \textbf{\textit{Rationale.}}

The use of ICD classification in genomic repositories imposes labeling often times based on clinical or empirical rather than pathophysiological characterizations of disease \cite{weiner_point_2018, cox_good_2009}. Use of these labels in polygenic risk score calculations poses the risk of concealing significant associations of polymorphisms with distinct sub-types of heart failure within the HFpEF classification \cite{severino_structural_2020}. Here we take an unsupervised clustering approach to identify clusters of patients within the HFpEF classification based on single nucleotide polymorphisms (SNPs). We then calculate the polygenic risk score of each cluster and compare them to the polygenic risk score calculated for the entire HFpEF population as a single cluster.

%%\indent \textbf{\textit{Preliminary Data.}} %%

\indent \textbf{\textit{Methods.}}

\indent \underline{Subaim 1a.} \textit{Identify subtypes of cardiovascular disease and heart failure phenotypes in UK BioBank and Mount Sinai BioMe using standard clustering and classification algorithms.} K-means clustering is an unsupervised machine learning technique which stratifies a dataset into k clusters based on the minimization of Euclidian distances in an n-dimensional space \cite{oyelade_clustering_2016}. We will perform K-means clustering on the set of patients identified by ICD-10-CM as having “diastolic heart failure,” also known as HFpEF. We will define our n-dimensional space as the set of all SNPs present across all HFpEF patients relative to GRCh38 as a reference genome \cite{ballouz_is_2019}. We will then proceed to compute polygenic risk scores for each of the k clusters of HFpEF patients.

%%[AAA] https://www.ncbi.nlm.nih.gov/pmc/articles/PMC5135122/
%%[AAB] https://genomebiology.biomedcentral.com/articles/10.1186/s13059-019-1774-4


\indent \underline{Subaim 1b.} \textit{Validate inferred phenotypic subtypes using SNP-level \cite{dahl_reverse_2019} and polygenic heterogeneity tests \cite{dahl_robust_2020} as well as using known sub-classifications of individual complex traits \cite{li_identification_2015} \cite{mordi_differential_2019}.} We use statistical methods to improve phenotype subtype inference and eliminate false positives. Regression methods including Multiple Traits with a Finite Mixture of Regressions (MFMR) \cite{dahl_reverse_2019} are used in subtype classification at the SNP and polygenic levels. MFMR considers a linear combination of covariate effects and discrete subtypes as contributions to a single quantitative trait $y_i$ for an individual $i$. We can model this combination as:

$$
y_i = X_i \alpha + \gamma_{z_i} + g_i \beta_{z_i} + \epsilon
$$

Genetic information in the form of principal components and any other covariates of interest are captured in $X_i$ with effect sizes $\alpha$. $z_i \in \{1,...,K\}$ specifies the subtype for an individual and is parametrized by $K$. $\gamma_{z_i}$ represents subtype effects on trait. $\beta$ captures subtype specific effect sizes from focal covariate $g$. $\epsilon$ is i.i.d Gaussian error term with mean zero. The MFMR multiple regression models traits for each individual as a function of trait heterogeneity from subtypes. In practice, we fit MFMR with an Expectation-Maximization algorithm and train with a range of $K$ specified possible subtypes for out trait of interest. We test MFMR with known cardiovascular disease subtypes \cite{mordi_differential_2019} and assess explained phenotypic variance $R^2$.  

%%expand upon these papers methods%%

\indent \underline{Subaim 1c.} \textit{Compare $\Psi$PRS explained phenotypic variance to standard genome-wide PRS.} An $R^2$ will be computed for each locus determined to have significant contribution for all PRS obtained from each k-means cluster of the HFpEF class as well as the PRS obtained from the entire HFpEF population. SNPs identified to significantly correlate with disease state will be manually compared and a literature review will be performed to assess the implication of each in disease. The SNPs which are identified by individual clusters of HFpEF but not in the aggregate HFpEF group will be of particular interest.

\indent \textbf{\textit{Benchmarking and expected outcomes.}} We expect to see K-means clustering stratifies the HFpEF population according to groups of SNPs that are implicated in specific co-morbidities (e.g. HFpEF patients with co-morbid hypertension will colocalize while HFpEF patients with co-morbid T2DM will colocalize, etc.) as well as cohorts of SNPs that are not particularly associated with a co-morbidity.

\hspace{10mm} The latter group of clusters are of particular interest because they represent genomic differences between HFpEF patients that may indicate a causal basis for heart failure. An excellent benchmark will be to perform correlation analyses of clinical metadata between each of these clusters. Assessments of average age of onset, disease progression, and various clinical indicators will be made and compared between each cluster to identify clinical features that correlate with the genotypic clusterings.



\indent \textbf{\textit{Anticipated problems and alternative approaches.}}
The greatest anticipated problem in our clustering approach is the known association of HFpEF with various co-morbidities such as hypertension, T2DM, and obesity \cite{ergatoudes_non-cardiac_2019}. As the latter conditions often have a strong genetic component, it seems evident that many clusters will obtain based on known SNPs in these conditions which are not inherently related to HFpEF but instead are confounds in identifying sets of SNPs associated with HFpEF. 

\hspace{10mm} It is possible to impose exclusion criteria of patients that present with HFpEF without major co-morbidities, but this is likely to considerably reduce the cohort size and compromise the statistical power of the polygenic risk score. An alternative approach is to use statistical techniques to account for the co-morbidities associated with HFpEF, such as by imposing biased weights to reduce the association of loci implicated in the co-morbid conditions. This would better isolate loci associated with HFpEF and not known co-morbidities.


% -- % -- % -- % -- % -- %
%% ancestry %%
\subsection*{Aim 2. Investigate reduced PRS transferability in non-European ancestry populations.}

\indent \textbf{\textit{Rationale.}} Most human genetic studies consider only European ancestry populations and as a result, most PRS constructed from genetic studies fail to accurately predict disease risk in non-European populations \cite{wojcik_genetic_2019}. This reduction in PRS transferability is a result of differences in population structure and in the underlying genetic etiologies of diseases of interest. Population geneticists refer to population structure as the combination of genetic variation, evolutionary processes including natural selection, mutation, recombination and differing demographic histories of distinct populations. Several complex diseases have been found to have statistically significant sub-classifications, but ICD-10 labeling fails to capture distinct etiologies in many cases \cite{weiner_point_2018, cox_good_2009}. It is likely that some disease genetic etiologies may be more common in some non-European populations than in European populations. In these situations, European-derived PRS may accurately capture genetic contributions to disease along one etiology but not along others. Therefore, PRS that consider the underlying mechanisms of population structure and heterogeneity in disease genetic etiology will more accurately explain phenotypic variance and improve transferability.         

%%\indent \textbf{\textit{Preliminary Data.}}%%

\indent \textbf{\textit{Methods.}} 

\indent \underline{Subaim 2a.} \textit{Model multi-way admixture of global sub-populations using simulations in standard \textbf{SLiM} population genetic software \cite{haller_slim_2019} and genotypes from the 1000 Genomes Project reference panels \cite{noauthor_global_2015}.} Simulations are commonly used in population genetics and genomic analyses to study the effects of population structure. We can control population parameters including natural selection, mutation, recombination and demographic histories with simulation software. We model global human genetic diversity with the \textbf{S}election on \textbf{Li}nked \textbf{M}utations (SLiM) simulation package \cite{haller_slim_2019}. SLiM provides broad flexibility in assessing the contributions of each of the aforementioned aspects of population structure on the observed outputs of typical GWAS: variant frequencies and phenotype effect sizes. We can importantly use SLiM and genetic data from the 1000 Genomes Project \cite{noauthor_global_2015} global human genetic database to simulate multi-way admixture. We can generate a set of progeny genotypes representing combinations of the 1000 Genomes Project continental reference panels (AFR, AMR, EAS, SAS, EUR). These simulations allow us to approximate human genetic diversity in non-European populations and investigate the specific elements of population structure that have the greatest impact on PRS transferability.

\hspace{10mm} We also impose sets of causal variants for a simulated trait of interest $y$. For instance, let set $\Omega$ contain all variants that contribute to the trait. If the trait has $K$ many subtypes with discrete genetic etiologies, then set $\beta_K \subseteq \Omega$ and contains the variants and effect sizes at which they convey trait $y$. We can specify effect sizes in simulations and allow causal variants to segregate within populations. Then, we can simulate a GWAS and recover causal variants and effect sizes from the simulated genotypes and trait statuses in SLiM. We can assess PRS from these estimated variant and effect sizes. Because we specify the causal variants and effect sizes \textit{a priori}, this approach serves as a control and validation of the predictive power of our PRS estimates when compared with known values. 

\indent \underline{Subaim 2b.} \textit{Compute ancestry-specific PRS with data from multi-ethnic populations in model along with Mount Sinai BioMe and UK BioBank to investigate causes of reduced transferability: differences in associated variant frequencies across populations, missing effects of population-specific rare variants and discrepancies in phenotype classification.} We build upon our previous subaim to assess the reliability of PRS in multi-ethnic populations. While the UK Biobank contains a limited representation of global genetic diversity, there remains some significant representation from non-European ancestry individuals within the database. The Mount Sinai BioMe Biobank contains information from a more diverse cohort of volunteers overall, but is smaller in absolute number of volunteers compared to the UK Biobank. We aim to combine these resources to assess PRS transferability. How will PRS computed specifically on the non-European ancestry individuals in UK Biobank fair when applied to the diverse populations of BioMe and vice versa? We propose a comparative analysis of PRS between the two resources, taking into account differences in sequencing coverage and phenotype labeling. Pooling these resources would offer important boosts to sample size and statistical power. 

\indent \textbf{\textit{Benchmarking and expected outcomes.}} We consider benchmarks that enable construction of robust models that capture the nuances of biological systems and population genetics. We report summary statistics to assess the strength of association, effect sizes and degree of genetic diversity. The $R^2$ for our PRS capture the variation in phenotype explained by our PRS. Wright's fixation index $F_{ST}$ \cite{wright_differential_1945} measures overall population differentiation in our multi-ethnic genomic data. We also report the relationship between $F_{ST}$ and PRS $R^2$, expecting as the fixation index increases (greater genetic diversity in our sample) the PRS $R^2$ decreases. Therefore, we expect increases in PRS $R^2$ when we construct ancestry-specific PRS on non-European populations with shared ancestry, because shared ancestry typically implies lower $F_{ST}$.         

\indent \textbf{\textit{Anticipated problems and alternative approaches.}} It is possible that our computed PRS will explain a small fraction of phenotypic variance ($R^2 < 0.05$). While such results are not uncommon in many reported PRS, we may choose to instead measure explained variance using a "pseudo-$R^2$" value \cite{choi_tutorial_2020}. This may be better suited for case/control outcomes, in essence, presence or absence of a clinical phenotype. Whereas a standard $R^2$ is better defined for continuous phenotypes such as height. However, pseudo-$R^2$ may be more susceptible to statistical bias in ascertainment. To cross-validate these measures, we can compare pseudo and standard $R^2$ using our simulations to ensure comparability. We also anticipate challenges in pooling UK Biobank and BioMe data, including discrepancies in sequencing reads and phenotype labeling. We will perform careful quality control to match information from both resources and take consideration of batch effects \cite{gilad_reanalysis_2015}.  


%%\textbf{\textit{Proposed timeline.}}%%




% -- % -- % -- % -- % -- %
\newpage
\bibliographystyle{unsrt} %unsrt should work, too.  copy myrefstyle.bst in the same directory as the .tex file.
\bibliography{references_zotero.bib, references.bib} % Or wherever you keep your .bib file.

% -- % -- % -- % -- % -- %
\end{document}

